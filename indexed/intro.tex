\documentclass[output=paper]{langsci/langscibook} 
\author{%
Silvia Hansen-Schirra\affiliation{Johannes Gutenberg University of Mainz in Germersheim}%
\lastand
Sambor Grucza \affiliation{University of Warsaw}
}

\title{Eyetracking and Applied Linguistics}

\ChapterDOI{10.17169/langsci.b108.232} %will be filled in at production


\abstract{}
\maketitle
\begin{document}
%\setcounter{page}{1}
\section{Introduction}

Eyetracking has become a powerful tool in scientific research and has finally found its way into disciplines such as Applied Linguistics and Translation Studies, paving the way for new insights and challenges in these fields. The aim of the first International Conference on Eyetracking and Applied Linguistics (ICEAL) was to bring together researchers who use eyetracking to empirically answer their research questions. It was intended to bridge the gaps between Applied Linguistics, Translation Studies, Cognitive Science and Computational Linguistics on the one hand and to further encourage innovative research methodologies and data triangulation on the other hand. Despite their inherent common interests, methodological exchange between these disciplines is rare. Thus, the conference served as a platform for interdisciplinary exchange and to exploit synergy effects. This proceedings volume focusses on the major topics that emerged during the aforementioned conference: \isi{audiovisual translation}, \isi{post-editing} as well as \isi{comprehensibility} and \isi{usability}. Eyetracking methodology is employed to empirically investigate all of said topics. 

The first part of the volume is dedicated to empirical studies in \isi{audiovisual translation}. The volume begins with a contribution by Wendy Fox, who tests the \isi{efficiency} of \isi{integrated titles} vs. traditional \isi{subtitles} for the language pair English-German. For the creation of the \isi{integrated titles}, she considers placement and design, which are proven to have an effect on reading time of the titles and perception of the image. The eyetracking data show that \isi{split attention} improves in favour of a more undistracted gaze behaviour of the image. Minako O’Hagan \& Ryoko Sasamoto present an eyetracking study on Japanese \isi{impact captions} intended to have an entertaining effect from the perspective of the TV producers. They contrast the differing results, which show unconscious eye movements vs. interviews focussing on the conscious impression of the subjects. As a result, the limitations and advantages of the research question as well as the methodology are discussed. Finally, Juha Lång introduces two experiments investigating the degree of \isi{information acquisition} from a subtitled television documentary for the language pair Russian-Finnish. The eyetracking data is triangulated with comprehension tests involving subject groups with different language skills. The author concludes with presenting the parallel processing of two information channels, i.e. \isi{narration} and \isi{subtitles}, as well as the \isi{efficiency} of the \isi{subtitles} and their potential for distraction.

The second part of this volume deals with \isi{post-editing} of \isi{machine translation} output. Within this context, Jean Nitzke focusses on \isi{monolingual post-editing} for English-German \isi{machine translation}. She investigates the quality of the final translations as well as the research patterns of the post-editors on the basis of the eyetracking data. She concludes with an evaluation of typical error types and the effort needed to accomplish the \isi{monolingual post-editing} tasks. Within the same realm, Fabio Alves and colleagues analyse the \isi{cognitive effort} exerted during \isi{post-editing} from a relevance-theoretical perspective. The concepts of conceptual and procedural \isi{encoding} are used for the empirical operationalization of the eyetracking results. These reveal the \isi{efficiency} of \isi{interactive post-editing} in contrast to other translation tasks and promote this type of \isi{computer-aided translation}.

\largerpage
Finally, this volume addresses questions of \isi{comprehensibility} and \isi{usability}. Christoph Rösener reports on the experiences of creating a \isi{usability} lab with eyetracking technology. He discusses general challenges and obstacles as well as specific equipment issues in concrete terms. He exemplifies his considerations by introducing the design of the \isi{\isi{usability} laboratory} at Flensburg University of Applied Sciences. Sascha Wolfer tackles the topic of \isi{comprehensibility} in Text Linguistics. In his eyetracking study, he contrasts German jurisdictional texts with reformulations intended to be more comprehensible with respect to nominalisations. The eyetracking \isi{corpus} is investigated in terms of reading times, regression paths as well as statistical probability assessments. While the reformulations can also be regarded within the paradigm of \isi{intralingual translation}, the eyetracking data focus on the readability and \isi{processing effort} for the given text type and thus on \isi{empirical research} in Applied Linguistics in general.

While the studies contained in this volume draw from a variety of objectives and various areas of overlaps between Applied Linguistics, Translation Studies and Cognitive Science, they all agree on eyetracking as an appropriate methodology in \isi{empirical research}. However, it should be emphasised that the volume is by no means exhaustive with regard to this research area. Further crossfertilisation is not only desirable, but almost mandatory in order to tackle future tasks and endeavours, and the ICEAL conference series remains committed to bringing these fields even closer together.

As a final remark, we would like to thank everyone who participated in making the ICEAL conference series as well as this volume possible. The authors of the individual articles put a lot of time and effort into their papers, and it was a pleasure working with them. We are also deeply indebted to our anonymous reviewers for their thorough and thought-provoking work, which decisively contributed to the quality of this volume.\\

\noindent Silvia Hansen-Schirra \& Sambor Grucza



%\subsection*{Abbreviations}
%\subsection*{Acknowledgements}

{\sloppy
\printbibliography[heading=subbibliography,notkeyword=this]
}
\end{document}