\title{Eyetracking and Applied Linguistics}    
\author{Silvia Hansen-Schirra\lastand   Sambor Grucza}
\renewcommand{\lsSpineAuthor}{S. Hansen-Schirra \& S. Grucza}
\BackTitle{Eyetracking and Applied Linguistics} % Change if BackTitle != Title
\BackBody{Eyetracking has become a powerful tool in scientific research and has finally found its way into disciplines such as Applied Linguistics and Translation Studies, paving the way for new insights and challenges in these fields.
The aim of the first International Conference on Eyetracking and Applied Linguistics (ICEAL) was to bring together researchers who use eyetracking to empirically answer their research questions. It was intended to bridge the gaps between Applied Linguistics, Translation Studies, Cognitive Science and Computational Linguistics on the one hand and to further encourage innovative research methodologies and data triangulation on the other hand. These challenges are also addressed in this proceedings volume: While the studies described in the volume deal with a wide range of topics, they all agree on eyetracking as an appropriate methodology in empirical research. }
%\dedication{Change dedication in localmetadata.tex}
\typesetter{Oliver \v{C}ulo, Sebastian Nordhoff, Florian Stuhlmann}
\proofreader{%
Alec Shaw,
Andreas Hölzl,
Anelia Stefanova,
Anne Kilgus,
Benedikt Singpiel,
Eitan Grossmann,
Gabrielle Hodge,
Georgy Krasovitskiy,
Jean Nitzke,
Joseph de Veaugh,
% Kristina Pelikan	
Martin Haspelmath,
Rachele De Felice,
Stathis Selimis,
Viola Wiegand
}
\BookDOI{10.17169/langsci.b108.230}%ask coordinator for DOI
\renewcommand{\lsISBNdigital}{978-3-944675-98-5}
\renewcommand{\lsISBNhardcover}{978-3-946234-65-4}
\renewcommand{\lsISBNsoftcover}{978-3-946234-69-2}
\renewcommand{\lsISBNsoftcoverUS}{978-1-537653-20-4}
\renewcommand{\lsSeries}{tmnlp}  
\renewcommand{\lsSeriesNumber}{2}
\renewcommand{\lsURL}{http://langsci-press.org//catalog/book/108} % contact the coordinator for the right number
\renewcommand{\lsAdditionalFontsImprint}{, Aozora Mincho}

%<*coverdimen>
\setlength{\csspine}{11.44mm} % Please calculate: Total Page Number (excluding cover), usually (Total Page - 3) * 0.0572008 mm
\setlength{\bodspine}{20mm} % Please use BoD's algorithm: http://www.bod.de/hilfe/coverberechnung.html (German only, please contact LangSci staff for help)
%</coverdimen>

 
